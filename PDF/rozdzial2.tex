\chapter{Implementacja}
\thispagestyle{chapterBeginStyle}

Przedmiotem tego rozdziału będzie opis implementacji trzech podprogramów realizujących: 
\begin{enumerate}
	\item aproksymacyjny algorytm First Fit Decreasing 
	\item sformułowanie liniowe z sekcji \ref{linear_formula} - dwie metody:
	\begin{enumerate}
		\item naiwna - releksacja IP do odpowiadającego LP (Linear Programming) i zaokrąglenie rozwiązania (algorytm simpleks i FFD)
		\item MIP - algorytm branch and cut
	\end{enumerate}
	 
\end{enumerate}

Wszystkie trzy zostały zaimplementowane w języku \textbf{C}, w standardzie C17. Motywem tego wyboru były prędkość wykonywania skompilowanego programu, jak i fakt, że API wybranego \textbf{solver GLPK} zostało napisane w C, co pozwala na bezpośrednie jego wywołania z poziomu tego języka\cite{GLPK_API}.
Zrealizowano je w postaci programu konsolowego, który dla odpowiednich flag wywołuje odpowiadające funkcje. Użyto GLPK (GNU Linear Programming Kit) w wersji 5.0. 
Procesem kompilacji zarządza narzędzie \textbf{Makefile}.
Struktura projektu zawartego na dołączonej płycie CD prezentuje się następująco:

\dirtree{%
	.1 SourceCode.
	.2 Makefile.
	.2 bin.
	.3 main.
	.2 obj.
	.2 src.
	.3 approx.c.
	.3 approx.h.
	.3 gen\_configs.c.
	.3 io\_handler.c.
	.3 io\_handler.h.
	.3 main.c.
	.3 solver\_part.c.
	.3 solver\_part.h.
	.3 structs.h.
	.3 vector.c.
	.3 vector.h.	
}

Program analizuje argumenty wiersza poleceń w stylu UNIX/Posix (funkcja getopt), stad systemy z nim zgodne powinny być obierane jako platforma docelowa kompilacji. Niezbędna też jest obecność plików nagłówkowych, statyczny, współdzielonych bibliotek solvera GLPK. Instrukcję jak zainstalować niezbędny pakiet moża znaleźć pod tym linkiem: \href{https://en.wikibooks.org/wiki/GLPK/Linux_OS}{instrukcja instalacji GLPK}.

\newpage
Dostępne argumenty wywołania wyświetlą się po wywołaniu programu z flagą h: ``\verb|SourceCode/bin/main -h|``, wtedy zostanie wyświetlony następujacy komunikat:

\begin{lstlisting}
	Avail flags:
	-f <in_filename.txt>
	-o <out_filename.txt>
	-a <ALGO>
		AVAIL ALGOS: 
		* APPROX 
		* MIP 
		* SM 
\end{lstlisting}


\section{Dane wejściowe} \label{format_danych}
Pierwszy punkt z przepływu danych w programie, czyli zczytanie wejścia można zrealizować na dwa sposoby. Pierwszy to ręczne wpisywanie z klawiatury w konsoli danych o które wnoszą komunikaty. Drugi to przygotowanie wcześniej pliku w odpowiednim formacie i podanie jego ścieżki jako argument do programu po fladze \textbf{-f}. Tak prezentuje się format danych dal obydwu przypadków: \\


\begin{table}[!h]
	\begin{center}
		\begin{tabular}{ p{5cm}p{2cm}p{7cm} }
			\multicolumn{3}{l}{$\beta$ (długość belki)} \\
			\multicolumn{3}{l}{$d$ (liczba różnych długości - rodzajów elementów)} \\
			$p_1$ (długość elementu 1) &  \texttt{\char32} & $n_1$ (potrzebna liczba sztuk elementu 1)\\
			$p_2$ &  \texttt{\char32}  & $n_2$ \\
			\vdots & \vdots & \vdots \\
			$p_d$ &  \texttt{\char32} & $n_d$ \\
		\end{tabular}
		\caption{Format danych wejściowych. Każdy wiersz to nowa linia, ``\texttt{\char32}`` oznacza białe znaki }
	\end{center}
\end{table}
Dane wejściowe są zapisywane do struktury \textit{ProblemInstance}, która zawiera w sobie tablicę sturktur typu \textit{ObjWithQuantity}(jej polami przechowują $p_i$, $n_i$). Ta druga sortowana jest według długości obiektów w kolejnośći nierosnącej algorytmem quicksort.
Każde ułożenie elementów na pojedyńczej belce jest reprezentowane przez stukturę typu \textit{StockConfig}. Zawiera ona tablicę, która informuje ile elementów danej długości, zaczynając od najdluższych przy zerowym indeksie, zostało upakowanych do belki. Druga zmienna w tej strukturze odpowiada pozostałej w belce przestrzeni po odjęciu długości zajmowanej przez konfigurację.

\subsection{Przykład}\label{example}
Za przykład posluży tutaj dobrze znany w literaturze zestaw danych o długości belki 5600 z trzynastoma rodzajami elementów \cite{EXAMPLE_REF}.

\begin{lstlisting} 
5600
13
1380 22
1520 25
1560 12
1710 14
1820 18
1880 18
1930 20
2000 10
2050 12
2100 14
2140 16
2150 18
2200 20 
\end{lstlisting}

\section{Algorytm aproksymacyjny}
Główna funkcjonalność ma miejsce w funkcji zdefiniowanej nastepująco: \\ \verb|size_t approx(ProblemInstance *input, Vector *v)| w pliku \textbf{SourceCode/src/approx.c}. \\
Jest to implementacja pseudokodu z sekcji \ref{approx_pseudocode}.
Składają się na nią głównie trzy zagnieżdżone w sobie pętle iteracyjne.
Pierwsza iteruje po rodzajach obiektów ($d$ razy).
Następna wewętrzna iteruje tyle razy ile wynosi liczba potrzebnych elementu danego typu ($n_i$).
Ostatnia, najgłębiej zagnieżdżona, przechodzi po dynamicznie zaalokowanej tablicy (zmienna typu \textit{Vector}) \textit{otwartych} belek, zaczynąjąc od tego do którego ostatniego mieścił się element danego typu. Pozwala to uniknąć niepotrzbnego sprawdzania belek, co do których jest pewność, że dany typ się nie zmieści. 
Po wyjściu z trzeciej pętli następuje sprawdzenie warunku, czy dany obiekt został umieszczony w którejś z otwartych belk. Jeśli tak się nie stało, tworzona i dodawana do dynamicznej tablicy zostaje nowa belka z wpisanym do niej, obecnie rozważanym do włożenia, elementem.

\section{Programowanie liniowe}
\subsection{Generowanie konfiguracji}
Problemem pośrednim w przypadku implementacji sformułowania liniowego z sekcji \ref{linear_formula} jest sposób generowania możliwych konfiguracji mogących wystąpić na pojedyńczej belce.
Z racji występowania w formułach znaków nierówności, a nie równości, pozwolono sobie na całkowite wypełnianie dostępnej w belkach przestrzeni, co nie wpływa jednak na finalną liczbę potrzebnych belek, a jedynie na pozostałą przestrzeń w każdej z nich, która zostanie podzielona pod nadmiarowe elementy. W optymalnym rozwiązaniu pozostałaby po prostu niepodzielona.\\
Kod znajduje się w pliku: \verb|SourceCode/src/gen_config.c|,  \\ zdefiniowany w funkcji: \\ \verb|static inline int gen_configs_tree(ProblemInstance *input, glp_prob *lp);| \\
Z powodu prostoty i efektywności zaimplementowany został algorytm generujacy konfiguracje leksykograficznie. Patrząc na jego schemat abstrakcyjnie, generuje on $d + 1$ - poziomowe drzewo, ktorego węzły reprezentują pozostałą przestrzeń, krawędzie ile elementów danego typu (poziomu), zaczynając od najdłuższego, mieści się w konfiguracji, która jest czytana jako wartości wezłów na ścieżce od korzenia (długośc belki) do liścia (pozostała finalnie przestrzeń).
Szczegółowy opis tej metody można przeczytać w pracy Saad M.A. Sulimana \textit{Pattern generating procedure for the cutting stock problem} \cite{GEN_CONFIGS}. \\

\subsection{Macierz ograniczeń} \label{macierz_ograniczen}
Do przechowywania wygenerowanych w opisany wyżej sposób danych, wykorzystane zostało wywołanie API solvera GLPK zdefiniowane następująco: \\
\verb|void glp_set_mat_col(glp_prob *P, int j, int len, const int *ind, const double *val)| \\
Z opisu w dokumentacji \cite{GLPK_DOCS} można dowiedzieć się, że:
\begin{itemize}
	\item \textbf{P*} - wskaźnik na obiekt będący instancją problemu
	\item \textbf{len} - $0 \leq len \leq m$, gdzie $m$ to długość $j$-tej kolumny, w naszym przypadku $m = d$
	\item \textbf{ind*} - wskażnik na tablicę indeksów, przykładowo, gdy ind[$i$] = $k$, wtedy val[$i$] będzie w $k$-tym wierszu $\mathbf{j}$-tej kolumny macierzy ograniczeń
	\item \textbf{val*} - wskażnik na tablicę wartośći
\end{itemize}
\textbf{val} i \textbf{ind} powinny mieć rozmiar len+1, ponieważ numeracja indeksów w bibliotece GLPK rozpoczyna się od jedynki, w przeciwieństwie do numeracji w języku C, rozpoczynającej się od zera. \\

Dla przytoczonego powyżej przykładu \ref{example} jedna z konfiguracji mogłaby wyglądać następująco:

\begin{lstlisting}
col.config to tablica: {0, 0, 0, 0, 0, 0, 1, 0, 2, 0, 0, 0, 0}            
// col (zmienna typu StockConfig)	
// zmienne przekazywane w wywołaniu bibliotecznym przezentowałyby się tak: 
ind[1] = 7; val[1] = col.config[1-1] // = 1; 
ind[2] = 9; val[2] = col.config[2-1] // = 2; 
\end{lstlisting}


Przy wielu instancjach PCB może okazać się, że finalna macierz ograniczeń ma postać macierzy rzadkiej, czyli takiej w której większość elementów ma wartość zero. Dlatego wartym wpomnienia jest fakt, że nie trzeba obawiać się wpisywania zer w tablicę val, ponieważ macierz ograniczeń w GLPK została zaimplementowana w taki sposób, iż zapamiętuje jedynie niezerowe wartośći, cytując dokumentację: \textit{Zero elements are allowed, but they are not stored in the constraint matrix}. To pozytywnie wpływa na zużycie pamięci operacyjnej jak również na prędkość wywoływanych z macierzą jako argumentem algorytmów. W kodzie źródłowym biblioteki można znaleźć opis implementacji tego mechanizmu \cite{GLPK_SRC}.

\subsection{Nakładanie ograniczeń}
Kolejnym krokiem w uzupełnieniu danych potrzebnych solverowi jest nałożenie ograniczeń na  wiersze i kolumny. W tym przypadku na $d$ wierszy zostały nałożone  ograniczenia dolne, ktore odpowiadają wartościom po prawej stronie znaków nierówności z formuły (wymaganej liczbie elementów danego typu). Kolumnom należało ustawić ograniczenie dolne równe zero.
Dla wierszy odpowiada za to funkcja: \\
\verb|void glp_set_row_bnds(glp_prob *P, int i, int type, double lb, double ub)|
\begin{itemize}
	\item \textbf{P*} - to samo co wcześniej
	\item \textbf{i} - numer wiersza na który nałożone mają zostać ograniczenia
	\item \textbf{type} - typ ograniczenia, nas interesuje wartość numeryczna GLP\_LO, czyli ograniczenie dolne, wtedy dla zmiennej pomocniczej związanej z $i$-tym wierszem wygląda to tak: $\mathbf{lb} \leq x \le +\infty$
	\item \textbf{lb} - wartość ograniczenia dolnego
	\item \textbf{ub} - wartość ograniczenia górnego
\end{itemize}
Analogicznie wygląda to dla kolumn: \\
\verb|void glp_set_col_bnds(glp prob *P, int j, int type, double lb, double ub)|

\subsection{Typ zmiennych}\label{typ_zmiennych}
Również w przypadku MIP należy dać znać solverowi od których zmiennych wymagamy, aby były całkowite. W naszym przypadku są to wszystkie, a jest ich tyle ile kolumn (konfiguracji).
Potrzebne jest wywołanie funkcji:
\verb|void glp_set_col_kind(glp_prob *P, int j, int kind)|

\begin{itemize}
	\item \textbf{P*} - to samo co wcześniej
	\item \textbf{j} - numer kolumny której rodzaj ma być zmieniony
	\item \textbf{kind} - rodzaj kolumny, wartość GLP\_IV odpowiada za typ całkowity
\end{itemize}

\subsection{Funkcja celu}
Kierunek optymalizaji (minimalizacja czy maksymalizacja) można ustawić wywołując funkcję: \\
\verb|void glp_set_obj_dir(glp_prob *P, int dir)| \\
\textbf{dir} jest w stanie przyjąć wartości kryjące sie pod stałymi: \\
GLP\_MIN - minimalizacja \\
GLP\_MAX - maksymalizacja \\

Za ustawienie współczynników $j$-tej zmiennej w funkcji celu odpowiada funkcja: \\
\verb|void glp_set_obj_coef(glp_prob *P, int j, double coef)| \\
W tym przypadku wszystkim zmiennym zostają nadane współczynniki równe jeden.

\subsection{Metody rozwiązujące problem}
{\parindent0pt
Plik \verb|Sourcode/src/solver_part.c|
}
\subsubsection{Naiwne podejście}
Pierwszym etapem jest usunięcie ze wzoru \ref{linear_formula} ograniczeń całkowitoliczbowych zmiennych. Wystarczy nie wywoływac metody glp\_set\_col\_kind i pozostawić domyślny typ zmiennych - ciągły. Następnym krokiem jest rozwiązanie powstałego w ten sposób LP (nazywanego relaksacją programowania liniowego) - do tego posłuży algorytm sympleksowy. Poniżej definicja funkcji, która go wywołuje:\\
\verb|int glp_simplex(glp_prob *P, const glp_smcp *parm)|. \\
Otrzymane w ten sposób rozwiązanie (zmienne) należy zaokrąglić do najbliższej wartośći całkowitoliczbowej. 
Oczywiście rozwiązanie otrzymane w ten sposób nie będzie optymalne, co więcej, może być niespełnialne, ponieważ może naruszać nałożone ograniczenia, konkretnie liczbę potrzebnych elementów danego rodzaju, może się okazać, że będzie ich za mało. Aby sobie z tym poradzić można przerzucić uzyskane w ten sposób konfiguracje do wektora (dynamicznej tablicy) belek, policzyć ile elementów danego rodzaju brakuje i z tymi dwiema informacjami wywołać algorytm approksymacyjny FFD, który je dopakuje.

\subsubsection{MIP}
Precyzyjnie rozpatrywany jest problem IP, szczególny przypadek MIP, którego wszystkie zmienne decyzyjne są dyskretne.
Dlatego też dokumentacja GLPK odnosi się tylko do MIP. Samo GLPK do rozwiązywania MIP używa solvera opartego na metodzie \textit{branch-and-cut}, która jest hybrydą metod \textit{branch-and-bound} i \textit{cutting plane}. \\
Fukcja wywołująca optymalizację MIP: \verb|int glp_intopt(glp_prob *P, const glp_iocp *parm)|. \\
Gdy w parametrach został wyłączony \textit{presolver}, co jest ustawieniem domyślnym, obiekt problemu (P) powinien zawierać optymalne rozwiązanie relaksacji LP, które można uzyskać rutyną wywołującą solver sympleksowy.

\textbf{DODAWAĆ COŚ O BRANCH AND CUT, BO NIE WIEM CZY WSTAWIANIE TU OPISU MA SENS???})


Wywołania metod mogą wyglądać podobnie jak niżej:
\begin{lstlisting}[language=C]
/* Początek wspólnej sekcja dla obydwu podejść */
// lp - wskaźnik na zmienną typu glp_prob
glp_smcp simplex_param; // zmienna odpowiadająca za parametry 
                        // metody sympleksowej
glp_init_smcp(&simplex_param);  // inicjacja domyślnych parametrów
glp_simplex(lp, &simplex_param); // wywołanie rutyny, która przeprowadza 
				// optymalizację algorytmem sympleksowym
	
/* Koniec sekcji wspólnej */
/* w przypadku metody naiwnej pozostaje zaokrąglić wyniki i wywołać FDD */

/* Sekcja odpowiedzialna za MIP */
for (size_t j = 1; j <= numOfCols; j++)
{
	glp_set_col_kind(lp, j, GLP_IV);
}
glp_iocp mip_parm;   // zmienna odpowiadająca za parametry MIP
glp_init_iocp(&mip_parm); // inicjacja domyślnych parametrów
mip_parm.presolve = GLP_OFF;
glp_intopt(lp, &mip_parm);
	
\end{lstlisting}


\subsection{Odzyskanie rozwiązania}
\subsubsection{Metoda sympleksowa}
{\parindent0pt
\verb|int glp_get_prim_stat(glp_prob *P)| - informuje o statusie podstawowego rozwiązania dla określonego obiektu problemu, wartość GLP\_FEAS - rozwiązanie jest spełnialne \\
}
\verb|double glp_get_obj_val(glp_prob *P)| - uzyskanie wartości funkcji celu (liczby potrzebnych belek)  \\
\verb|double glp_get_col_prim(glp_prob *P, int j)| - zwraca wartość zmiennej związanej z j-tą kolumną (liczby potrzebnych konfiguracji o indeksie j) - interesują nas wartości większe od zera, które należy zaokraglić

\subsubsection{Metoda MIP}
Funkjce zwracają wartości analogicznie jak powyżej. \\
\verb|int glp_mip_status(glp_prob *P)| - informuje o statusie rozwiązania dla wywołania metody MIP, wartość GLP\_OPT - rozwiązanie jest optymalne pod względem liczby całkowitej \\
\verb|double glp_mip_obj_val(glp_prob *P)|\\
\verb|double glp_mip_col_val(glp_prob *P, int j)| - pomimo typu \textit{double} zwracane wartości są całkowite i nie ma potrzeby ich zaokrąglania, wystarczy je zrzutować np. na \textit{size\_t}

Dla obydwu metod pozyskanie konfiguracji wygląda tak samo: \\
\verb|int glp_get_mat_col(glp_prob *P, int j, int ind[], double val[])| - zwraca dwie tablice, które mają strukturę analogiczną jak w przypadku umieszczania konfiguracji w instancji problemu (patrz \ref{macierz_ograniczen}). \\


Po odpowienim sformatowaniu rozwiązania uzyskanego powyższymi metodami (w przypadku podejścia naiwnego również zaokrągleniu, uzupełnieniu brakujacych elementów algorytmem FFD) można wydrukować je na wyjście (terminal lub plik).

Tekst wyjścia programu dla przytoczonego wyżej przykładu można zobaczyć na obrazie \ref{output_example}. Każda konfuguracja C[$i$] o indeksie $i$ składa się z $d$ (w tym wypadku trzynastu) kolumn, odpowiadających posortowanym rodzajom obiektów po słowie LENGHT i QUANTITY. Wartość każdej z tych kolumn to liczba elementów danego typu występująca w konfiguracji. 
W przypadku MIP występują dodatkowe, rozpoczynające się znakami: $\%$, x, po których nastepują wartości odpowiedno: pozostałej niewykorzystanej przestrzeni w pojedyńczej belce, liczbie potrzebnych belek z daną konfiguracją.
Gdy w programie wykonywana była metoda APPROX, każda belka jest rozpisywana osobno, w nowym wierszu. Nie pojawia sie kolumna informująca ile razy dana konfiguracja została użyta.
Dodatkowo przed konfiguracjami pojawiają się informacje takie jak wybrany algorytm, liczba kolumn (wygenerowanych konfiguracji), rozmiar macierzy, liczba całkowitych zmiennych, status rozwiązania etc. 

\textbf{Pomocną w trakcie testów będzie informacja o ograniczeniu dolnym \\ (LOWER\_STOCK\_NUM\_BOUND) - jest to suma długości wszystkich elementów podzielona przez długość belki i zaokrąglona w górę - nie istnieją poprawne rozwiązania o mniejszej liczbie belek. Stąd, tak jak w tym przypadku, gdy dolna granica jest równa liczbie belek z uzyskanego rozwiązania, możemy mieć pewność, że jest ono optymalne pod tym względem.}

\begin{figure}[h!]
	\lstinputlisting{example_result.txt}
	\caption{Przykładowe wydrukowane wyjście dla programu używajacego MIP}
	\label{output_example}
\end{figure}







