%Korekta ALD - nienumerowany wstęp
%\chapter{Wstęp}
\addcontentsline{toc}{chapter}{Wstęp}
\chapter*{Wstęp}

\thispagestyle{chapterBeginStyle}



Praca swoim zakresem obejmuje implementację programu rozwiązujacego problem cięcia belek.

Celem pracy jest zaprojektowanie i oprogramowanie aplikacji o następujących założeniach funkcjonalnych:
\begin{itemize}
    \item wspieranie przedsiębiorstw w optymalizacji kosztów produkcyjnych
\end{itemize}

Istnieje szereg aplikacji o zbliżonej funkcjonalności: np. gotowe opisanie problemu w solverze CPLEX, zamodelowanie problemu w pythonie na macierzach, przy czym albo są to rozwiązania wymagjące wiedzy technicznej od użytkownika.

Praca składa się z czterech rozdziałów.
W rozdziale pierwszym omówiono skąd pomysł na zajęcie się tym problemem i jak można wynikowe programy wykorzystać.

{\color{dgray}
W rozdziale drugim przedstawiono szczegółowy projekt systemy w notacji UML. Wykorzystano diagramy \ldots.
Opisano w pseudokodzie i omówiono algorytmy generowania danych potrzebnych do zamodelowania problemu.

W rozdziale trzecim opisano technologie implementacji projektu: wybrany język programowania, biblioteki. Przedstawiono dokumentację techniczną kodów źródłowych interfejsów poszczególnych modułów systemu.

W rozdziale czwartym przedstawiono sposób instalacji i wdrożenia systemu w środowisku docelowym.

Końcowy rozdział stanowi podsumowanie uzyskanych wyników.
}

