\begin{abstract}
The purpose of this paper was to consider a variant of the Cutting Stock Problem in which the number of lengths is fixed. Possible approaches to the problem in theoretical terms were presented: approximation algorithms and a linear (integer) formulation of the problem. \textit{A Polynomial Time OPT + 1 Algorithm for the Cutting Stock Problem with a Constant Number of Object Lengths} \cite{ALG_OPT_1} was shortly described. However, due to the lack of methods used by the algorithm in the available solvers and its high degree of complexity, it was not implemented in this project. Therefore, in the practical part, the focus is on the description, programming procedures using, in turn, the First-Fit-Decreasing approximation algorithm, the naive approach to integer programming (relaxation, rounding the results returned by the simplex method) and the exact solution of the integer problem using a solver based on the\textit{branch-and-cut}. The paper compares them on sample data sets. It turned out that in most cases the solutions obtained using approximation algorithms had a cost that did not differ by more than 1 from the optimal cost. 
\end{abstract}

