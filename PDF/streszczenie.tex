\begin{streszczenie}
Istniejące algorytmy, które rozwiązują ogólnie sformułowany Problem Cięcia Belek albo działają w czasie wielomianowym, ale  dzieje się to dla małych danych wejściowych i/lub zwracają koszt daleki od optymalnego albo uzyskują koszt optymalny, ale działają w czasie wykładniczym. Pewną metodą na poradzenie sobie z tym, może być skupienie się na specyficznym odgałęzieniu problemu, jak w przypadku wielomianowego algorytmu OPT+1 dla stałej liczby długości. Dla instacji problemu ze stałą liczbą długości udowodnione zostało zwracanie przez niego wyniku maksymalnie o jeden większego od optymalnego, zachowując przy tym złożoność wielomianową\cite{ALG_OPT_1}. Teoretyczne rozważania prezentuje się bardzo dobrze. Niestety, wykorzystywana przez niego metoda elipsoidalna nie doczekała się implementacji w powszechnie dostępnych solverach liniowych. Warto jednak jest przedstawić przeszkody z którym próbuje on sobie radzić, jak i również to czy algorytmy, które pesymistycznie zwracają wyniki dalekie od optymalnych, dla przykładowych, prostych danych testowych nie dadzą zadowalających rezulatów.
\end{streszczenie}
