\begin{streszczenie}

Celem pracy było przyjrzenie się odmianie Problemu Cięcia Belek, w którym liczba długości jest stała. Przedstawiono  w ujęciu teoretycznym możliwe podejścia do problemu: algorytmy aproksymacyjne i sformułowanie problemu liniowego (całkowitoliczbowego). Opisano w skrócie
\textit{Algorytm OPT+1 dla Problemu Cięcia Belek ze Stałą Liczbą Długości}\cite{ALG_OPT_1}. Jednak ze względu na brak implementacji w dostępnych solverach oraz dużego skomplikowania algorytmu, nie został on w niniejszym projekcie wdrożony. Dlatego w części praktyczniej skupiono się na opisaniu, zaprogramowaniu procedur, które wykorzystują kolejno: algorytm aproksymacyjny First-Fit-Decreasing, naiwne podejście do programowania całkowitoliczbowego (realaksacja ograniczeń, zaokrąglanie wyników zwracanych przez metodę sympleksów), oraz dokładne rozwiązywanie problemu całkowitoliczbowego solverem bazującym na algorytmie \textit{branch-and-cut}. W pracy porównano je na przykładowych zbiorach danych. Okazało się, że w większości przypadków rozwiązania uzyskane algorytmami przybliżającymi posiadały koszt nie różniący się o więcej niż 1 od kosztu optymalnego. 
	
\end{streszczenie}
